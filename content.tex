
\proptitle{\fulltitle}

\head{Project overview}

\subhead{JD big-picture (fake subhead)}

Information related to health risks and healthy behaviour is typically generated by scientists and distilled into recommendations by public-health agencies. From there it is often transmitted by mass media to the public. Misinformation, generated by careless or irresponsible scientists, pseudo-scientists, or product-marketers can follow a similar route. In the age of the internet, these routes remain important, but the public also has easy direct access to information from public-health agencies, and to many scientific papers, including unvetted preprints. The public also has an expanded ability to interpret, transmit, and amplify messages through social media. More recently still, scientists and agencies have become active on social media as well.

During disease outbreaks of international concern, like pH1N1, WAEO ((fill in better names with dates)) or the current COVID-19 outbreak, this process is both compressed and amplified. The stakes also become higher, because public behaviour directly affects the spread of infectious disease: people who avoid large gatherings may slow disease spread, while people who flee infected areas may accelerate it, for example. Excessive fear of disease spread can have severe economic effects, and may also lead to bias and discrimination against groups seen as linked to the disease, or merely seen as others.

JD: Can we work on the diagram and this \P\ in parallel?
\mlicomment{Can we try to give a clearer defintion of ``information'' or examples of information?}
The proposed research will study how information flows between forums, including scientific publications; governmental policies and agency recommendations; and mass and social media -- and investigate how it affects public perceptions and behaviours. 
Our inter-disciplinary group will analyze these flows by combining textual and contextual analysis; AI-assisted human-supervised data mining; time-series analysis; and mathematical modeling. We will study how good information competes with misinformation, and look for factors correlated with successful spread of good information. 
We will gather information on communication and surrogates for behaviour from a wide range of sources. 
Sources about communication will include agency websites; preprint servers and publicly available scientific journals; major newspaper websites; social-media platforms; and twitter data.
Surrogates for public perceptions and behaviour will include twitter data (again); google trends; publicly available records for movie majors box office receipts; publicly available travel data; and information about cancellations and shortages (for example of face masks or pharmaceutical or pseudo-pharmaceutical products) from our textual analysis. 

\subhead{Research Question}
Our research questions were framed based on the object of developing potential social and policy countermeasures for COVID-19 outbreak (under the CIHR special call object).

\subsubhead{How does information (and misinformation) travel between scientists, public-health workers, mass media and social media?}
\begin{itemize}
\item{How does media cover and disseminate information?}
\item{How media content are consumed among the general public, science community and the national and international govern units?}
\end{itemize}

The aim of this question is to congregate contents to address the social and cultural dimensions of different population during the epidemic such as, examining how individuals and communities understand and react to the disease.

\subsubhead{How does communication affect public behaviour and the course of the outbreak?} 
\mlicomment{This question needs to be sharper. It is too vague and can mean different things to different people in its current form.}

This research question studies and compares the public health institutes response and guidelines and how policy affect public health behavior in reacting to the infection?
(CS: by fitting data from google trends, news and twitter and public health authority)


\subsubhead{How can scientists and policy-makers evaluate and improve the effectiveness of their communication?}
This research question aim to develop strategies to combat misinformation, stigma, and fear; and to improve public awareness and knowledge.

(CS: by comparing policy, public reaction and effectiveness (reported by public health authority and by media))

The proposed research will improved existing infomation flow by (1) curating the infomation, (2) understand how interest groups are interpreting and communication information. 
The proposed research will work to bridge the gap between theoretical advances in epidemic modelling inference, public health practice and .. with interest groups. 

\head{Background}

\subhead{Generation facts about COVID-19}
COVID-19 is a coronavirus (CoV), that is pathogenic to humans, and the ability to directly transmit between humans has posed a significant threat and numbers compared to the previous coronavirus outbreaks: severe acute respiratory syndrome (SARS) coronavirus (2002-SARS-CoV) and Middle East respiratory syndrome (MERS) coronavirus (2012-MERS-CoV). 
Public health officials have reacted quickly in response to the outbreak; many researchers in the scientific community have published their preliminary analyses of the outbreak as pre-prints to estimate potential risk. 
\mlicomment{how are media/public responding to the news/preliminary analysis from the science folks? Below is how the scientist are responding.}
However, there are a lot of uncertainties in both modeling and effective control responses that can cause miscommunication between modellers and public and greatly affect control.   
\mlicomment{What do we know that is true?}
Without an effective vaccine available in time, the current control strategy is strictly quarantine and isolation and much are still unknown how effective these countermeasures are in controling the spread of the disease.  

\subhead{Media and Infectious Disease}
\mlicomment{This is where I think C's text goes.}

Success of curtailing disease outbreak depends on not only public health system, but also public adherence based on their perception of the disease.  Media have influence in swinging public perception of health issues and their estimation of health risk by framing what to present and in what context and in affecting their risk estimation, for example, overestimating rish during SARS outbreak \citep{BerrWalf07} or triggering vaccinating panic on a transmission dynamics of influenza \citep{TuchDube11}.  It can also be biased against “others” as Western newspapers portrayed Chinese during the SARS crisis \citep{HuanLeun06} (https://doi.org/10.1080/01292980500261621).  Media is also the tool public health authorities rely on to promote their concerns and recommendations during health risk outbreak.  Understanding media effect on disease spread (e.g., media attention increases self-protection) can help enhance epidemic forecasting and preventive measures to slow the disease spread \citep{KimFast19}.  

While news media (due to its online virtue) still carry their influence, social media is shaping how we communicate and understand information \citep{LiuSieg19}.  When social media undoubtedly provided benefits for public health promotion and disease prevention \cite{BascHill20, SunYang20}, for example, people using twitter for health information were merely likely to be vaccinated \citep{AhmeQuin18},  it also impose potential threats, such as discrimination, disinformation and misinformation \citep{ChouOa18, McKevanS19}.  Since the initial reports of cluster of acute severe respiratory disease (COVID-19) and the potential for global spread, there has been widespread discussion and dissemination of information through social media \citep{?}.  Twitter is a real-world disease outbreak-reporting source, and could report case outbreak ahead of official reports. \citep{YoutDara19}.  It can be used to circulate information from credible sources but also as a source of experiences and opinions \citep{ChewEyes10}.  The nature of some of the information and conversations on social media could be discriminatory and increase panic about the risk of infection. The feeling of anxiety in the population, in part can be assessed by increased demands and shortages for face masks and hand sanitizers even in countries such as Canada where only x cases confirmed to date. 

(CS:  we need a few setences about policy makers and google trend impact and roles to tie this all together)

\mlicomment{maybe make a plot using google trend for the keyword search ``coronavirus'' and HuBei time series}

\head{Methods and feasibility}

\subhead{Data and Sources}

Data of media dated from November 2019 (before when it was officially acknowledged to ensure no news about this was overlooked) till June 2020, when the outbreak is expected to temper down, will be collected.  

Newspaper with top circulation and at least two formats (print and online) with online database will be collected from Singapore, Taiwan, Hong Kong, China, Canada and USA and England. The first four will be mainly in Chinese and the latter three in English.  Our plan to study newspaper in England because it carries readers in Euruope where is also affected by the outbreak.  Singapore, Taiwan and H.K. are chosen because of its cultural and social ties with China and the virus impacts they confronted.  With exception under the criteria, we plan to gather also Sing Tao Daily(https://www.singtao.ca/toronto/?variant=zh-hk), the most circulated Chinese news paper in Canada; and World Journal(https://www.worldjournal.com/) in the United States in attempt to assess cultural and social dimension in the news content among Chinese immigration population in North America.  We will use Lexis-Nexis search engine to extract news.  OriProbe Information Services( https://www.oriprobe.com/peoplesdaily.shtml ) provides access to the archive of People’s Daily of China.  Key words used for searching is attached (Becky’s figure) 

Twitter data (countries and region covered?) will be collected by (python or AI or purchased? need some details on how to access tweets. ) We will apply AI techniques to Twitter stream data based on the coding book to analyze tweets and parse out themes and to apply algorithm to identify false information through topic modelling.  We will also track evolution of information content and varsity over time and across geographic areas.
	
GoogleTrends accumulate data of search frequency around the world by region and county and by time.  It reflects awareness-related burst of searches \cite{BousAgac17, MahrBrag19}.  We use it as an index to reflect the concern and interest the public expressed during the outbreak.  key words search for Google trend, and official public health websites (discussed below) include coronavirus, 19-nCov, COVID-19, influenza, 武疫 (Wu Virus),武漢病毒 (Wuhan Virus),武漢(Wuhan),武漢肺炎 (Wuhan pneumonia),肺炎 (pneumonia),冠状病毒 (coronavirus), 新型冠状病毒 (novel coronavirus), 武漢新型冠状病毒 (Wuhan novel coronavirus).(CS:  someone fill in how we will collect and analyze the trends data)

The epidemic findings from science community is part of the information the public, media and international community rely on. They play a powerful role during public health crises due to the time urgency with which they can disseminate new information, accurate or not. (ref: [Early in the Epidemic: Impact of preprints on global discourse of 2019-nCoV transmissibility](https://papers.ssrn.com/sol3/papers.cfm?abstract_id=3536663)).  We can also check findings of preprint (speedy information delivery, lack of peer review) in terms of accuracy and how misinformation get circulated based on those findings. (see [examples](https://papers.ssrn.com/sol3/papers.cfm?abstract_id=3536663))

Science publications of COVID-19 (peer-reviewed and preprint) will be extracted by searching Google Scholar, arXiv, bioRxiv, medRxiv, and SSRN, , PubMed, Embase, Medline, and OVID).  We will construct themes of the studies and results.  Our team will evaluate the impact of the publications and compare to whether it is cited and twitte/re-twitte by the other forums.  

Official health websites:  WHO, the official centres of disease control and prevention in the studied countries (e.g., US CDC , National Health Commission of PRC (http://en.nhc.gov.cn/index.html), PHAC, Canada. 
 
\subhead{Analysis of Data}
Textual analysis using AI is reliable and efficient in terms of data mining.  However, it can be insufficient in understanding what contents connote.  For example, meanings of shortage of face masks in Toronto is likely different from the shortage in Hong Kong heavily impacted with COVID-19 by cased infected.  Without referring to circumstance, we will miss what the results signify and its impact on the public.  To amend this issue, we will conduct contextual analysis \citep{} to elaborate the results of textual analysis of tweets and news and uncover its social and cultural perspectives of the text meanings.

At the final stage of our data analysis,  we will plot google trends data, the daily infected cases and fatality and frequency and main themes of media coverage and tweets,  we expect to construct a proxy indicator for information dissemination and public reaction to the outbreak.  Furthermore, we hope to uncover the communication flow among the four forums.


\mlicomment{MLi's stuff}

\subsubhead{HuBei Province}
Daily updates from the Provincal Health Commission of HuBei are avalible online.
These updates contains information on up-to-date cumulative cases, new cases per day, number of confirm case hospitalized, severity categorizations and deaths for all cities in HuBei province.

\subsubhead{Surveillance}
Daily updates from outside of HuBei are avalible online.
These updates contains information on up-to-date cumulative cases, new cases per day, number of confirm case, severity categorizations and deaths for different locations.

\mlicomment{CFS's stuff}

(ref theme analysis in the H1N1 paper, and PMID: 31956275) Data published from November, 2019 to present (or one or two month after the outbreak waned)

\subsubhead{Media}
Newspaper will be collected from country of Singapore, Taiwan, Hong Kong, Japan (?), Canada and USA (and England?) based on the top circulation (either in English or Chinese)
\begin{itemize}
\item{USA: Associated Press (AP) newswire; U.S. English-language newspapers of top high-circulation; most viewed youtube (and broadcast news transcripts from top networks?).  Lexis-Nexis will be used to collect texts from the first three sources, and the MIT MediaCloud database to collect texts from websites. Texts were collected from both sources using broad search terms citep{LiuSieg19} ; and [World Journal](https://www.worldjournal.com/)}
\item{Canada: (similar to USA), and [Sing Tao Daily](https://www.singtao.ca/toronto/?variant=zh-hk)}
\item{Taiwan: the top x newspapers. https://www.ncl.edu.tw/}
\item{Hong Kong: [Epoch Times](https://www.epochtimes.com/gb/news415.htm)}
\item{China:  [The People’s Daily English](http://en.people.cn/), [The People’s Daily English](http://www.people.com.cn/), [China Daily](http://global.chinadaily.com.cn/)}
\item{Singapore ??}
\item{Japan: [Japanese Times](https://www.japantimes.co.jp/)}
\item{Twitter: (CS:  How does it work internationally outside Canada and USA?).  It will be great if we can have social media (instgram or FB in Taiwan, Singapore etc)}
\item{Main newspapers in China and their official health bureau websites are included for analysis but not social media.   Social media such as twitter and Facebook are barred in China. Yet, there are information and coverage about the outbreak inside China in the media we analyze either from direct interview or re-twittes. }
\end{itemize}

\cscomment{twitter methods \citep{ChewEyes10, LiuSieg19, YoutDara19}}

\subsubhead{Google Trend data}
\begin{itemize}
\item{Google Trends provides search around the world by region and county.  Research showed that Google Trend can be influenced by the media attraction than by true epidemiological burden \citep{}, yet more evidences suggested that it reflects awareness-related burst of searches citep{BousAgac17, MahrBrag19}}
\item{By analyzing google trends data, the daily infected cases and fatality and frequency of media coverage,  we use Google Search Trends and media coverage as a proxy indicator for information dissemination and public reaction  
%(https://papers.ssrn.com/sol3/papers.cfm?abstract_id=3536663).
}
\item{e.g. 
%https://trends.google.com/trends/explore?date=today%201-m&geo=SG&q=%2Fm%2F01cpyy,wuhan%20virus,19-nCov,sars,flu ,([export google trend data]( https://support.google.com/trends/answer/4365538?hl=en)
}
\item{Google Trend data can be very confusing, for [example]
%(https://trends.google.com/trends/explore?date=today%201-m&geo=TW&gprop=images&q=coronavirus,sars,%2Fm%2F0l3cy,wuhan%20virus,%E6%AD%A6%E6%BC%A2%E8%82%BA%E7%82%8E)): 
Why would Wuhan topped coronavirus and ``(WuHan pneumonia)'' in Taiwan?
}
\end{itemize}

\subsubhead{Science community report of the virus transmission}
\begin{itemize}
\item{Science publication: preprint (by searching Google Scholar, arXiv, bioRxiv, medRxiv, and SSRN) and peer-reviewed (by search Google Schoolar, PubMed, Embase, Medline, and OVID)}
\item{The epidemic findings from science community is part of the information the public, media and international community rely on. They play a powerful role during public health crises due to the time urgency with which they can disseminate new information, accurate or not. (ref: [Early in the Epidemic: Impact of preprints on global discourse of 2019-nCoV transmissibility]
%(https://papers.ssrn.com/sol3/papers.cfm?abstract_id=3536663)).  
We can also check findings of preprint (speedy information delivery, lack of peer review) in terms of accuracy and how misinformation get circulated based on those findings. (see [examples]
%(https://papers.ssrn.com/sol3/papers.cfm?abstract_id=3536663))
}
\item{(CS’s confusion: Is context same as noise in modelling (e.g., Park and Dushoff 2020):  dynamical or process; noise(randomness directly or indirectly affecting disease transmission); observation noise(randomness underlying how many true cases reported).  Based on their definition, how do context or media, public reaction and pubic policy-guideline fit into models?  Is it a right question?)
}
\item{(CS: It will be good to incorporate modelling ideas in this project but WHAT?  Also, how workable is it to compare this coronavirus to 2012 MERS and 2002 SARS in terms of the spread and reaction context ?)
}
\end{itemize}

\subsubhead{Official websites to analyze social and policy countermeasures}
How does the national and international community react to the disease outbreak and with what guidelines (analyzing official websites, such as WHO, CDC, Chinese Center for Disease Control and Prevention, etc)

\subsubhead{Official health websites}
\begin{itemize}
\item{WHO, [US CDC]
%(https://www.cdc.gov/coronavirus/index.html)
}
\item{[National Health Commission of PRC
%(http://en.nhc.gov.cn/index.html)
}
\item{Canada PHAC
%(https://www.canada.ca/en/public-health/services/diseases/2019-novel-coronavirus-infection.html)
}
\item{Taiwan CDC
%(https://www.cdc.gov.tw/En)
}
\item{Singapore CDC
%(https://www.ncid.sg/Pages/default.aspx)
}
\item{[Hong Kong CDC]}
\end{itemize}
We will include the disease control and prevention centre in Japan later if ??.

We will add sources (e.g., Central China Normal Univeristy, Wuhan University
%(https://www.whu.edu.cn/xxfy/) 
etc for our contextual analysis.


We will need to curate the data-stream.

\subhead{Data mining}

textual analysis of news media and Twitter are implemented by AI based on coding book which will be based on a pilot study. 
The findings of textual analysis Human Coders and AI (?):
\begin{itemize}
\item{(fill in what it is about )}
\item{( fill in sentences on AI part)}
\item{(fill in sentences on coders)}
\item{The textual analysis will be conducted based on a coding book we will draft when the funding is approved. The coding book should list themes and frames for analyzing the texts, which will be categorized based on a pilot study as part of drawing the coding book. The basic structure of the themes will outline what (e.g. ), in addition to information types (e.g., news, information and misinformation/fake news based on sources: verified, anonymous, speculation), who (e.g. active vs. passive subject), where (e.g., country, region and city), and theme tone (e.g., negative, positive, neutral). (see attached Basic Coding frame for what themes are like)
}
\end{itemize}

\subhead{Analysis}

\subsubhead{Contextual analysis by researchers}
\begin{itemize}
\item{(fill in what contextual analysis is)}
\item{How media content and information are framed are often socially and politically cultivated \citep{}. We will implement contextual analyses of the quantitative findings and elaborate how information is framed and disseminated in this uncertain circumstances into context.}
\item{summarizing and comparing policy, guidelines and recommendation by national and international health organizations.}
\end{itemize}

\subsubhead{Modelling}
(Can we do this?: incorporating media coverage, public and institutional reaction and report (which can be based on modelling results by scientific community and WHO etc), time travelling and spread network (geographical)

\begin{itemize}
\item{key words search for Google trend analysis, and official public health websites: coronavirus, 19-nCov, COVID-19, influenza, (Wu Virus), (Wuhan Virus), (Wuhan), (Wuhan pneumonia), (pneumonia), (coronavirus), (novel coronavirus), (Wuhan novel coronavirus)}
\end{itemize}

\subsubhead{general public}
\begin{itemize}
\item{Data: travel, movies, restaurants in Taiwan, Singapore, H.K., Canada and USA; or should we focus on North America and on travel and movie box office?}
\end{itemize}

\subsubhead{deliverables}

\begin{itemize}
\item{software}
\item{Communication: Writing papers is certainly one, but we should be tweeting and etc. What are communication platforms we should be communicating in?}
\end{itemize}


\head{Research Setting \& Personnel}

The principal applicants have a long history of influential research on disease outbreak: 

The research will principally take place at McMaster University. 
Nominated principal applicant Dr.\ David J.D. \textbf{Earn} (6 hours per week) led the creation of the International Infectious Disease Data Archive [33] and has expertise in gathering and curating infectious disease data, and in dynamical modelling.
Principal applicant Dr.\ Jonathan \textbf{Dushoff} (5 hours per week) is an internationally recognized expert in infectious disease modelling, has extensive experience with statistical frameworks for fitting models to data, and has been involved in the Ebola challenge and other forecasting projects. 
Co-Applicant Dr.\ Benjamin \textbf{Bolker} (5 hours per week) is a highly accomplished ecological statistician with extensive experience in spatial-dynamical modeling, statistical modeling and statistical software.
Co-Applicant Dr.\ Chyun-Fung \textbf{Shi} (40 hours per week), post-doctoral researcher, has ...
Co-Applicant Dr.\ Michael \textbf{Li} (5 hours per week), post-doctoral researcher, has focused his research on epidemic forecasting and is experienced working with large databases. 

\mlicomment{fill in the rest later}

\subhead{Collaborators and Knowledge Users}

\mlicomment{fill in later}

\head{Research Time line}

\subhead{Year 1}

\subhead{Year 2}

\subhead{Potential Outcomes}
Transfer findings to peer-reviewed publications.  
In addition to a grand one paper, we plan to transfer findings into subgroups, such as by affected country (North America, Asia- Taiwan, H.K. and Singapore), types of media (news media and social media (twitter)).

\begin{itemize}
\item{Potential to contribute to the global response to COVID-19}
\item{Social and policy countermeasures and Global Coordination Mechanism}
\end{itemize}

\head{Challenges and Mitigation Strategies}

Social media in China is to included due to data availability.   Social media such as twitter and Facebook are barred in China and WeChat does not share their database. Yet, there are information and coverage about the outbreak inside China in the media.  We will categorize people’s reaction and understanding of COVID-19  on the news content or the tweets to supplement our results.


\subsubhead{Data curation} 

We will go back and document clearly all the policy changes and case defintions. 

\begin{itemize}
\item{data collected from National Health Commission}
\item{Figure out how to use the data effectively (e.g. we are not using death, severity categorizations, number of tested, number of positive, and etc)}
\item{Case definitions}
\item{Media content}
\item{Social media platforms}
\item{Google trends}
\end{itemize}

\subsubhead{Analysis/ Pipelineing/ mainstreaming}

Info delays, misinformation and miscommunications

\subsubhead{Communication}

how people are interpreting 

\head{Conclusion/summary}

\subsection{Note}
We want to emphasize our bilingual (English and Chinese) background, which is essential in this study.



